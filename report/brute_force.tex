\subsection{Brute-force}

\subsubsection{Intuition}
The brute-force approach tries all possible combination of bridges to find the solution.
It guarantees correctness, but consumes great amount of time.

\subsubsection{Implementation}

\begin{enumerate}
  \item \textbf{Generate variables:} Define Boolean variables for all bridges.

  \item \textbf{Add logical constraints:}
  \begin{itemize}
    \item Prevent double bridges.
    \item Ensure that the total number of bridges connected to each island matches its degree.
    \item Forbid crossing bridges.
  \end{itemize}

  \item \textbf{Iterate through all assignments:}
  Use exhaustive enumeration to try every possible combination of \texttt{True} or \texttt{False} for all variables.

  \item \textbf{Constraint checking:}
  For each assignment, check whether all logical clauses are satisfied. If any clause is unsatisfied, skip to the next assignment.

  \item \textbf{Check connectivity:}
  If the assignment satisfies all clauses, check whether the resulting graph is connected.

  \item \textbf{Display result or fail:}
  If a valid combination is found, display the solution and stop. If after all assignments but there's no solution -> this puzzle is unsolvable.
\end{enumerate}