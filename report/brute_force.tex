\subsection{Brute-force}

\subsubsection{Intuition}
The brute-force approach exhaustively tries every possible combination of bridges — meaning, every possible truth assignment (True/False) to the Boolean variables that represent whether a bridge exists or not. For each assignment, it checks whether all the logical constraints are satisfied and whether the resulting graph is fully connected. If a valid assignment is found, the solution is displayed. Otherwise, the algorithm reports that no solution exists.

This approach guarantees correctness but is extremely inefficient for large instances because it explores all $2^n$ possible combinations, where $n$ is the number of variables.

\subsubsection{Implementation}

\begin{enumerate}
  \item \textbf{Generate variables:} Define Boolean variables representing all possible bridges between islands.

  \item \textbf{Add logical constraints:}
  \begin{itemize}
    \item Disallow double bridges between the same pair of islands.
    \item Ensure that the total number of bridges connected to each island matches its required degree.
    \item Forbid crossing bridges.
  \end{itemize}

  \item \textbf{Iterate through all assignments:}
  Use exhaustive enumeration to try every possible combination of \texttt{True} or \texttt{False} for all variables (i.e., every possible bridge configuration).

  \item \textbf{Constraint checking:}
  For each assignment, check whether all logical clauses are satisfied. If any clause is unsatisfied, skip to the next assignment.

  \item \textbf{Check connectivity:}
  If the assignment satisfies all clauses, verify whether the resulting graph is fully connected — all islands must be reachable from one another.

  \item \textbf{Display result or fail:}
  If a valid and connected configuration is found, display the solution and stop. If all $2^n$ assignments are checked without success, report that no solution exists.
\end{enumerate}