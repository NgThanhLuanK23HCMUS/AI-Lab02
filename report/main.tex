\documentclass[12pt, a4paper]{article} 
\usepackage[utf8]{inputenc}
\usepackage[english]{babel}
\usepackage{graphicx}
\usepackage{amsmath,amssymb}
\usepackage{lmodern}
\usepackage{tikz}
\usepackage{tabularx}
\usepackage{makecell}
\usetikzlibrary{arrows.meta, positioning}
\usepackage{iftex}
\ifPDFTeX
  \usepackage[T1]{fontenc}
  \usepackage[utf8]{inputenc}
  \usepackage{textcomp} % provide euro and other symbols
\else % if luatex or xetex
  \usepackage{unicode-math}
  \defaultfontfeatures{Scale=MatchLowercase}
  \defaultfontfeatures[\rmfamily]{Ligatures=TeX,Scale=1}
\fi
\usepackage{amsmath, amssymb, amsfonts}
\usepackage[left=2.5cm, right=2.5cm, top=2.5cm, bottom=2.5cm]{geometry}
\usepackage{caption} 
\usepackage{booktabs}
\usepackage{hyperref}
\hypersetup{
    colorlinks=true,
    linkcolor=blue,
    filecolor=magenta,
    urlcolor=cyan,
}
\usepackage{fancyhdr}
\usepackage{enumitem} 
\usepackage{lipsum}
\usepackage{float}
\usetikzlibrary{positioning, arrows.meta}

\pagestyle{fancy}
\fancyhf{} 
\fancyhead[L]{\nouppercase{\rightmark}}
\fancyhead[R]{\thepage} 
\fancyfoot[C]{}
\renewcommand{\headrulewidth}{0.4pt} 
\renewcommand{\footrulewidth}{0pt} 

% Trang bia
\title{
    \textbf{Ho Chi Minh University of Science} \\
    \textbf{Department of Information Technology} \\
    \vspace{0.5cm}
    \includegraphics[width=0.5\textwidth]{img/logohcmus.png} \\
    \vspace{1cm}
    \huge \textbf{Lab02 Report: Hashiwokakero} \\
    \vspace{1cm}   
    \Large \textbf{Course: Introduction to Artificial Intelligence} \\
    \Large \textbf{Instructor: Vo Nhat Tan} \\
}
\author{
    \shortstack[l] {
        \textbf{23127296 - Nguyen Thanh Luan} \\
        \textbf{23127302 - Tran Quang Phuc} \\
        \textbf{23127539 - Nguyen Thanh Tien} \\
        \textbf{23127543 - Vu Van Vu} \\
    }
}
\date{July, 2025} 

\begin{document}

\maketitle 

% muc luc
\newpage 
\tableofcontents 

% introduction
\newpage 

\section{Introduction}
\begin{table}[htbp]
    \centering
    \caption{Contribution table}
    \label{tab:contribution}
    \begin{tabular}{@{}lc@{}} % bo khoang trong
        \toprule
        \textbf{Work} & \textbf{Member} \\
        \midrule
        Formulate CNF constraints & Nguyen Thanh Luan\\
        Automate CNF generation & Nguyen Thanh Luan \\
        Design input files & Tran Quang Phuc \\
        Implement backtracking & Nguyen Thanh Tien \\
        Implement brute-force & Vu Van Vu  \\
        Implement A* & Tran Quang Phuc \\
        Solve using PySat & Nguyen Thanh Luan \\
        \bottomrule
    \end{tabular}
\end{table}

\begin{table}[htbp]
    \centering
    \caption{Requirement completion table}
    \label{tab:contribution}
    \begin{tabular}{@{}lc@{}} % bo khoang trong
        \toprule
        \textbf{Requirements} & \textbf{Completion level} \\
        \midrule
        Design at least 10 input files & Yes\\
        Formulate CNF constraints & Yes \\
        Automate CNF generation & Yes \\
        Use PySat to solve & Yes\\
        Implement backtracking & Yes\\
        Implement brute-force & Yes\\
        Implement A* & Yes\\
        Detailed explanation of algorithms & Yes \\
        Results and evaluation & Yes \\
        Report & Yes\\
        \bottomrule
    \end{tabular}
\end{table}
\section{Problem Introduction}
Hashiwokakero (also known as Bridges) is a Japanese puzzle. It is played on a rectangular grid. There are islands on the grid. The player can draw bridges between the islands in a straight line. The goal is to make all the islands form a single connected group. The puzzle follows the following rules:
\begin{itemize}
    \item They must begin and end at distinct islands, travelling a straight line in between.
    \item They must not cross any other bridges or islands.
    \item They may only run orthogonally (i.e. they may not run diagonally).
    \item At most two bridges connect a pair of islands.
    \item The number of bridges connected to each island must match the number on that island.
    \item The bridges must connect the islands into a single connected group.
\end{itemize}
\section{Program flow}
\section{Encode Hashiwokakero to CNF}
\subsection{Logical Variable Definition}
To obtain the CNF formula for the Hashiwokakero puzzle, first we introduce a logical variable \\
\[ x_{i,j,i',j',k} \]
where 
\begin{itemize}
  \item $(i,j)$ and $(i',j')$ is the location of the 2 islands.
  \item $k$ is the number of bridges between 2 islands.($k=1,2$)
\end{itemize}
\subsection{CNF constraints}
The following constraints are converted to CNF in order to solve the Hashiwokakero puzzle:
\subsubsection{Bridges must begin and end at distinct islands, travelling in a straight line}
\begin{itemize}
    \item The endpoints $(i,j)$ and $(i',j')$ must be distinct: $(i,j) \ne (i',j')$.
    \item The bridge must lie on a straight line: either the same row or the same column, therefore we can have
    \[
    \forall i, j, i', j', k \quad \left( x_{i,j,i',j',k} \rightarrow \left[ (i = i' \land j \ne j') \lor (j = j' \land i \ne i') \right] \right)
    \]
    
\end{itemize}
\subsubsection{Bridges must not cross any other bridges}
Two bridges $x_{i_1,j_1,i_1',j_1',k_1}$ and $x_{i_2,j_2,i_2',j_2',k_2}$ intersect if:

\begin{itemize}
  \item $i_1 = i_1'$, $j_2 = j_2'$ (one runs horizontal, one runs vertical),
  \item $i_2 < i_1 < i_2'$, and $j_1 < j_2 < j_1'$ (they cross each other),
  \item $\{i_1,j_1\}, \{i_1',j_1'\} \ne \{i_2,j_2\}, \{i_2',j_2'\}$ (they don't run from same islands).
\end{itemize}
\subsubsection{At most two bridges connect a pair of islands}
The number of bridges is equal or smaller than 2:
\[
    \forall i, j, i', j', k \quad ( x_{i,j,i',j',k} \rightarrow 0 \leq k \leq 2)
    \]
\subsubsection{The number of bridges connected to each island must match the number on that island}
Let $B_{i,j}$ be the number written on the island located at row $i$, column $j$. Let the binary variable $x_{i,j,i',j',k} \in \{0,1\}$ represent whether there is a bridge of type $k$ (where $k = 1$ for a single bridge and $k = 2$ for a double bridge) connecting island $(i,j)$ to island $(i',j')$.

The constraint can be written as:
\[
\sum_{\substack{i',j',k \\ (i,j,i',j',k) \in D}} k \cdot x_{i,j,i',j',k} = B_{i,j}
\]

Here, $D$ is the set of all valid pairs of islands $(i',j')$ that can be connected directly to $(i,j)$ (either horizontally or vertically, but not diagonally), and without crossing over another island.

More explicitly:
\[
\sum_{\substack{(i',j') \text{ such that } (i = i' \land j \ne j') \\ \text{or } (j = j' \land i \ne i')}} \sum_{k=1}^{2} k \cdot x_{i,j,i',j',k} = B_{i,j}
\]


\section{Algorithm Explanation}

\subsection{Backtracking}

\subsubsection{Intuition}
Backtracking is a trial-and-error technique for solving constraint problems by incrementally building a solution and checking at each step whether the current partial solution is valid. If at some point the current path cannot lead to a valid solution, the algorithm backtracks to try a different choice. In the context of the Hashiwokakero puzzle, backtracking will try assigning \texttt{True} or \texttt{False} to each bridge variable, partially checking CNF clause satisfaction, and ensuring that the final graph is fully connected.

\subsubsection{Implementation}
\begin{enumerate}
  \item \textbf{Generate variables:} Create Boolean variables representing possible bridges between islands.

  \item \textbf{Add logical constraints:}
  \begin{itemize}
    \item Ensure no double connections (i.e., a bridge is not added twice in the same direction).
    \item Enforce the degree constraints — each island must have the exact number of bridge connections as specified.
    \item Prevent bridges from crossing each other.
  \end{itemize}

  \item \textbf{Initialize the assignment:} 
  Create an assignment list to track which variables are currently set to \texttt{True} (bridge exists), \texttt{False} (bridge does not exist), or \texttt{None} (not yet assigned).

  \item \textbf{Partial constraint checking:} 
  During backtracking, after assigning a value to a variable, check whether all fully-assigned clauses are still satisfied. This prevents exploring invalid branches early.

  \item \textbf{Recursive backtracking:}
  Try assigning \texttt{True} or \texttt{False} to each variable in order.
  \begin{itemize}
    \item If the partial assignment does not violate any constraints, proceed to assign the next variable.
    \item If a violation is detected, undo the assignment and try the opposite value.
    \item If all variables are assigned, check if the resulting graph is fully connected.
  \end{itemize}

  \item \textbf{Solution or failure:}
  If a complete valid assignment is found and the graph is connected, the solution is displayed.
  Otherwise, the algorithm concludes that no valid solution exists.
\end{enumerate}
\subsection{Brute-force}

\subsubsection{Intuition}
The brute-force approach exhaustively tries every possible combination of bridges — meaning, every possible truth assignment (True/False) to the Boolean variables that represent whether a bridge exists or not. For each assignment, it checks whether all the logical constraints are satisfied and whether the resulting graph is fully connected. If a valid assignment is found, the solution is displayed. Otherwise, the algorithm reports that no solution exists.

This approach guarantees correctness but is extremely inefficient for large instances because it explores all $2^n$ possible combinations, where $n$ is the number of variables.

\subsubsection{Implementation}

\begin{enumerate}
  \item \textbf{Generate variables:} Define Boolean variables representing all possible bridges between islands.

  \item \textbf{Add logical constraints:}
  \begin{itemize}
    \item Disallow double bridges between the same pair of islands.
    \item Ensure that the total number of bridges connected to each island matches its required degree.
    \item Forbid crossing bridges.
  \end{itemize}

  \item \textbf{Iterate through all assignments:}
  Use exhaustive enumeration to try every possible combination of \texttt{True} or \texttt{False} for all variables (i.e., every possible bridge configuration).

  \item \textbf{Constraint checking:}
  For each assignment, check whether all logical clauses are satisfied. If any clause is unsatisfied, skip to the next assignment.

  \item \textbf{Check connectivity:}
  If the assignment satisfies all clauses, verify whether the resulting graph is fully connected — all islands must be reachable from one another.

  \item \textbf{Display result or fail:}
  If a valid and connected configuration is found, display the solution and stop. If all $2^n$ assignments are checked without success, report that no solution exists.
\end{enumerate}
\subsection{A*}

\subsubsection{Intuition}
A* is a best-first search algorithm that explores the most promising partial assignments first, guided by a cost function.

\subsubsection{Heuristics}
To guide the search efficiently, A* uses a heuristic function $h(n)$ that estimates the "cost to goal" from a given partial assignment. The total cost function is defined as:
\[
f(n) = g(n) + h(n)
\]
where:
\begin{itemize}
  \item $g(n)$ is the actual number of variables assigned so far (the path cost).
  \item $h(n)$ is the estimated number of unsatisfied clauses.
\end{itemize}

Several heuristics can be used:

\begin{itemize}
  \item \textbf{Clause-based heuristic:} Count the number of clauses that are currently unsatisfied under the partial assignment.

  \item \textbf{Optimistic heuristic:} Count only clauses that are fully assigned and violated. This makes the heuristic admissible and allows for more exploration.

  \item \textbf{Graph-based penalty (optional):} Add a penalty for partial solutions that look disconnected or under-connected, so the algorithm can be more likely to have globally connected configurations.
\end{itemize}

\subsubsection{Implementation}

\begin{enumerate}
  \item \textbf{Generate variables:} Represent all possible bridges between islands using Boolean variables.

  \item \textbf{Add constraints:} Encode logical conditions into CNF clauses:
  \begin{itemize}
    \item No duplicate bridges.
    \item Degree of each island must match its number.
    \item No crossing bridges.
  \end{itemize}

  \item \textbf{Initialize priority queue:} Start with an empty assignment and compute its $f(n)$ value.

  \item \textbf{Search loop:}
  \begin{enumerate}
    \item Pop the assignment with the smallest $f(n)$ from the priority queue.
    \item If all variables are assigned and all clauses are satisfied, and the resulting graph is connected, return the solution.
    \item Otherwise, pick one unassigned variable and expand the current state by assigning it both \texttt{True} and \texttt{False}.
    \item For each expanded state, compute $f(n)$ and push it into the priority queue.
    \item Optionally skip states that already violate a clause under full assignment.
  \end{enumerate}

  \item \textbf{Fail case:} If the priority queue is exhausted without finding a solution, report failure.
\end{enumerate}


\section{Results}
\subsection{Test cases}
\subsection{Results}
\subsection{Algorithm Comparisons}

\end{document}