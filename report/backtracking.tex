\subsection{Backtracking}

\subsubsection{Intuition}
Backtracking is a trial-and-error technique. If at some point the current path cannot lead to a valid solution, the algorithm backtracks.

\subsubsection{Implementation}
\begin{enumerate}
  \item \textbf{Generate variables:} Assign Boolean variables to bridges between islands.

  \item \textbf{Add logical constraints:}
  \begin{itemize}
    \item Ensure no double connections 
    \item Enforce the degree constraints.
    \item Prevent bridges from crossing each other.
  \end{itemize}

  \item \textbf{Initialize the assignment:} 
  Create an assignment list to track which variables are currently set to \texttt{True} (bridge exists), \texttt{False} (bridge does not exist), or \texttt{None} (not yet assigned).

  \item \textbf{Partial constraint checking:} 
  During backtracking, after assigning a value to a variable, check whether all fully-assigned clauses are still satisfied. This prevents exploring invalid branches.

  \item \textbf{Recursive backtracking:}
  Try assigning \texttt{True} or \texttt{False} to each variable in order.
  \begin{itemize}
    \item If the partial assignment does not violate any constraints, proceed to assign the next variable.
    \item If a violation is detected, undo the assignment and try the opposite value.
    \item If all variables are assigned, check if the resulting graph is fully connected.
  \end{itemize}

  \item \textbf{Solution or failure:}
  If a complete valid assignment is found and the graph is connected, the solution is displayed.
  Otherwise, the algorithm concludes that no valid solution exists.
\end{enumerate}